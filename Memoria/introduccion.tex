\chapter*{Introducción}\addcontentsline{toc}{chapter}{Introducción}

	Ya desde hace varias décadas, se planteaba la posibilidad de que las máquinas fuesen capaces de realizar tareas diferentes a meros cálculos. La persona que realizó dicha afirmación fue la matemática Ada Lovelace, que en 1842 programó el considerado primer algoritmo. No fue hasta bastantes años más tarde en 1956 cuando se celebra la Conferencia de Darmouth por los hoy considerados padres de la \gls{ia}, en la que se propone estudiar la Inteligencia Artificial como una ciencia más. En dicha época existían dos corrientes, la simbólica y la conexionista.\\
	
	La primera de ellas se ocupaba de resolver problemas de toma de decisiones y de obtención de conclusiones. Fueron populares los algoritmos de búsqueda y los sistemas expertos. Por otro lado, la corriente conexionista trataba de simular el comportamiento de las neuronas humanas de manera artificial, haciendo que estas neuronas artificiales pudiesen aprender. De aquí surgía el perceptrón que también fue presentado en esta conferencia. Más tarde, entre 1970 y 1980, con el libro de Minsky sobre los perceptrones y sus limitaciones, investigaciones en falso, y bajos recursos, decae el interés y la investigación de la \gls{ia}. Sin embargo, esta etapa vacía finaliza con la llegada del algoritmo de retropropagación que permitía entrenar redes neuronales multicapa, trayendo consigo infinidad de nuevos proyectos. Años después, avanzados los 2000, se empieza a ver lo poderosos que pueden llegar a ser ciertos modelos de \gls{ia} al vencer a campeones del mundo en juegos, como a Kasparov en el ajedrez o a Jie en Go. Con la llegada de más y mejores recursos y financiación, hacia 2010 se populariza el uso de redes neuronales para trabajar con imágenes, resolver problemas de clasificación, etc. \\
	
	Hace diez años surge uno de los modelos de \gls{ia} con el que se inicia el paradigma que más popularidad está tomando en la actualidad. Este es el de la \gls{ia} generativa. En 2014 Ian Goodfellow presenta las redes \gls{gan} con las que crear nuevos datos \cite{historiaIA}, por ejemplo rostros humanos que no existen, tal y como se muestra en \url{https://thispersondoesnotexist.com/}. Gracias a la \gls{ia} generativa, han ido surgiendo en los últimos años herramientas muy poderosas como ChatGPT con la que entablar cualquier tipo de conversación, solicitar información o ayuda para resolver cualquier problema; DALL-E o MidJourney a las que solicitar crear una imagen mediante una descripción por texto, o incluso un vídeo con Sora. \\
	
	Al intentar resolver un problema relacionado con el aprendizaje automático, el primer paso es elegir un modelo que se adapte correctamente al dominio y tipo del problema. De manera ingenua, se puede entender como una especie de caja negra a la que dada un conjunto de entradas devuelve otro conjunto de salidas que dependen de dichas entradas, mediante una serie de operaciones con respecto a un conjunto de parámetros $\Theta$. Por tanto, para obtener las salidas deseadas para una serie de entradas, el trabajo es encontrar los parámetros óptimos $\Theta^*$ que produzcan dichas salidas. Esto se logra mediante un algoritmo de aprendizaje o entrenamiento, siendo el segundo paso elegir uno acorde al modelo. Normalmente se dispone de dos tipos de aprendizaje, aprendizaje supervisado y aprendizaje no supervisado. Como tercer paso, se debe de medir de alguna manera cómo de bien o mal se está comportando el modelo y el algoritmo, tal y como se estudiará más adelante \cite{Szeliski}. \\
	
	En los problemas de aprendizaje supervisado, se dispone de un conjunto de datos o \textit{dataset}, que contiene los valores de salida deseados para diferentes valores de entrada, normalmente recogiendo situaciones del pasado para poder extrapolar este conocimiento a situaciones del futuro. Los principales problemas que de aprendizaje supervisado son los problemas de clasificación y de regresión. En los problemas de clasificación, se dispone de una serie de clases $C_1, C_2, \hdots, C_n$, y para una serie de valores de entrada $x_1, x_2, \hdots, x_m$, debe decidirse a qué clase pertenece dicha entrada. Un ejemplo sería decidir si un paciente va a sufrir un cierto tipo de cáncer dada su edad, peso, y otras constantes vitales. Algunos de los modelos más populares para llevar a cabo este tipo de tareas son árboles de decisión, máquinas de soporte vectorial, Naïve Bayes, $k-$vecinos, y redes neuronales; siendo estas últimas objeto de estudio en este trabajo. Otro tipo de problema popular a la hora de disponer de datos etiquetados, son los problemas de regresión, que se diferencian principalmente de la clasificación en que en este caso, los valores no son clases (valores discretos) sino valores continuos. Para resolver este tipo de problemas se suelen utilizar regresiones lineales y no lineales (exponencial, polinómica, etc), o también se pueden adaptar modelos utilizados en problemas de clasificación, como las redes neuronales. Un ejemplo de un problema de regresión sería predecir las horas que dormirá una persona dada su edad, horas trabajadas en el día, etc. \\
	
	Por otro lado, en los problemas de aprendizaje no supervisado no se dispone de los valores de salida esperados para una cierta observación (justo al contrario que en el caso supervisado), pues será trabajo del algoritmo encontrar relaciones y patrones entre los datos proporcionados. En este tipo de aprendizaje también se trata el problema de clasificación, sin embargo, es más común llamarlo clustering o segmentación, pues a priori no se conoce el número de clases y cuáles son, es el algoritmo el que deberá encontrar relaciones entre los datos para determinar esto. Algoritmos populares para realizar esta tarea son $k-$medias, \gls{cja}, \gls{gmm}, y \gls{dbscan}. Un ejemplo sencillo de este problema es detectar las diferentes regiones y objetos representados en una imagen, pues inicialmente no se conoce el número de regiones u objetos, y deben detectarse todas, asignando cada píxel de la imagen a cada una de ellas. \\
	
	En este trabajo, tras analizar diversos modelos de aprendizaje automático junto con sus características y algoritmos asociados en un marco teórico, se pretende aplicarlos en un caso práctico que aborda un problema de la vida real al que aún no se ha propuesto ninguna solución. Para ello, se presenta la compañía Niantic, fundada en 2010 como parte de una startup de Google, que se especializa en el desarrollo de videojuegos para dispositivos móviles que utilizan \gls{ar}. Algunas de sus creaciones más destacadas han sido los juegos Ingress y Pokémon GO. \\
	
	Una de las herramientas creadas por esta empresa es Niantic Lightship, que permite a desarrolladores Unity integrar realidad aumentada y mapas con puntos de interés basados en la ubicación real del jugador. Dado que para Niantic resultaba inviable marcar dichos puntos de interés alrededor de todo el mundo, creó Niantic Wayfarer. En esta herramienta, usuarios experimentados de sus juegos pueden hacer propuestas de puntos de interés (llamados Wayspots) para que de manera colaborativa, otros usuarios las valoren. Sin embargo, tras varios años desde su lanzamiento, debido al gran número de propuestas y al reducido número de valoradores, la comunidad notifica largos tiempos de espera en el proceso de valoración de las propuestas. \\
	
	En conclusión, como objetivo general de este trabajo se propone realizar una primera aproximación a la automatización de este proceso mediante el estudio de diferentes técnicas de \gls{nlp}, visión e inteligencia artificial. Como objetivos específicos, se proponen los siguientes. 
	
	\begin{itemize}
		\item Clasificación de una imagen en un conjunto cerrado de clases, dependiendo del objeto que aparece en esta. 
		\item Clasificación de una imagen en un conjunto abierto de clases, dependiendo del objeto que aparece en esta. 
		\item Búsqueda de imágenes dentro de un conjunto, mediante una descripción textual de estas. 
		\item Selección del título o descripción más adecuado para una imagen. 
		\item Aproximación a la detección de imágenes que contienen un mismo objeto. 
	\end{itemize}